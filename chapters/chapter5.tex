\chapter{Idealized WRF Study}


% from something  I wrote
\section{Methods}

In this modeling study, observations from the Norwegian Young Sea Ice Experiment (N-ICE2015) were used to validate the Weather Research and Forecasting model. First the model was run from 1 January 2015 to 1 July 2015 for each combination of the selected microphysics and boundary layer schemes. Next, three case study periods were selected for further study and analysis using idealized model runs using soundings taken at the field experiment site for input.

\subsection{Model Setup}

The Weather Research and Forecasting model (WRF) version 4.1.4 was run with polar optimizations created by The Ohio State University. These optimizations include improvements of heat transfer over ice and updating sea ice fraction (Hines and Bromwich 2008). These improvements have been tested over Arctic land and have had analysis done using SHEBA data, but have not yet been tested over first-year sea ice. The unique and large storms seen during N-ICE, along with the persistent surface temperature inversions present during the winter, make this dataset particularly interesting to use for Polar WRF validation.

A summary of the settings used can be seen in Table 1. A wide range of settings for both the microphysics and boundary layer parameters were selected primarily based on frequency of use in published studies. A range of boundary-layer parameters were selected to observe which most accurately represented the strong near-surface inversions in the Arctic and the radiative influence this can have on the surface. Alternatively, clouds also have strong radiative importance in the Arctic; cloud microphysics settings were selected for study here to determine which parameterization most accurately simulates these clouds, which are likely to be mixed phase or have supercooled water present. All other schemes, with the exception of the surface layer scheme, were kept constant throughout all simulations. The surface layer scheme was adjusted for use with the MYJ boundary layer from the Revised MM5 scheme to the ETA similarity scheme. Three combinations are absent from this analysis as they resulted in the model crashing, these are the WRF Single-Moment 5-Class microphysics and YSU boundary layer, the Predicted Particle (P3) microphysics and the MYNN boundary layer, and the Morrison Bulk 2-Moment microphysics and YSU boundary layer scheme.

A 3-nested domain setup was used and can be seen in figure 1. The highest resolution domain is 3km by 3km, located just north of Svalbard and encompassing the entire N-ICE field experiment domain. ERA Interim was used for boundary and initial conditions, PIOMAS was used for snow depth, ice thickness, and albedo, and SSMI was input for ice extent information. (Graham et al. 2017) includes a comparison of the ERA Interim dataset and the measurements taken during N-ICE. While it was found in this paper that ERA Interim accurately portrayed the cloudy and clear states, there were still issues with the cloud liquid water path being underestimated (Graham et al. 2017), which should be taken into consideration when looking at the WRF results. The model was run in two parts, a winter segment, including January through March, and a spring segment, including April through June. Comparison with the measurements does not start until January 29 and April 24th for the winter and spring runs, respectively, resulting in 20+ days of spin up time before analysis. The runs were completed using the Cheyenne supercomputer.

Idealized runs were completed using Pacific Northwest National Lab’s supercomputer Constance. ERA5 was used for both surface and large scale forcing and the input sounding used temperature, relative humidity, and wind values from the twice-daily soundings taken at the Norwegian Young Sea Ice experiment. Each idealized simulation was completed for one day before and after the day of interest to ensure proper spin up time. 

Case studies selected for idealized simulations and further study were selected based on times with the least disagreement between the input data and the experimental data. Two winter cases were selected to eliminate the shortwave component of the energy budget and remove any potential errors from that. However, one spring case study was selected as it was the only day during the entire experiment that clear sky was observed over the ship for the entire 24 hour period. Clear sky periods were of particular interest as past modeling studies [SOURCE] found that models had difficulty representing the strong inversion that often forms under clear sky in the Arctic.

\subsection{Field Experiment}
The Norwegian Young Sea Ice Experiment (N-ICE2015) was a 6-month long field experiment that measured atmospheric conditions from January to June 2015 (Granskog et al. 2016). Most notable for the analysis presented here are the atmospheric radiation measurements, taken by Kipp and Zonen (CMP22 and CGR4) radiometers at 1 to 1.2 m above the ground. (Granskog et al. 2016; Walden et al. 2017) includes a complete analysis of the radiative fluxes during the experiment and includes a description of how surface temperatures were calculated. Walden et al., 2017 showed that the radiative fluxes during N-ICE were primarily influenced by wind, advection, and cloud cover.

 The research vessel R/V Lance was frozen into first-year sea ice in the Atlantic sector of the Arctic ocean and drifted with the sea ice. Three times during the experiment the ice surrounding the ship broke up and the ship needed to be repositioned into the sea ice. The time it took for the ship to reposition can be seen as gaps in the data from February 21st through 24th, March 15th through April 24th, and again from June 5th through the 7th. Note that the period in March and April corresponds with a trip back to Svalbard, explaining the long duration of the data gap.

This field experiment is of particular interest from a modeling perspective due to the magnitude of the temperature and pressure change during and after winter storm periods. This experiment is the first to take measurements of the clouds and atmosphere since the Surface Heat Budget of the Arctic (SHEBA) field experiment in 1997 and 1998. However, the storms observed during the N-ICE2015 observational period showed temperature increases significantly larger than those seen at SHEBA (Cohen et al. 2017). Additionally, throughout much of the field experiment, strong temperature inversions were observed over the surface, similar to what was seen at SHEBA (Kayser et al. 2017). While these strong inversions are not unique to N-ICE2015, they are often underrepresented in Polar WRF simulations (Hines et al. 2015).

\section{Results}
\subsection{Case 1 - A Winter Cold Frontal Passage, 5 February 2015}

A cold front passed through the ship location just after midnight on 5 February, bringing a temperature drop of approximately 30 degrees C. Cohen et al., 2017, which defined major and minor storms during the N-ICE according to pressure and wind changes, indicated this was the second major storm that passed over the ship during the experiment period. More information about the synoptic setup can be found in Cohen et al., 2017 and details about the surface energy budget can be found in Walden et al., 2017. 

\subsection{Case 2 - A Winter Clear-Sky Period, 6 March 2015}

\subsection{Case 3 - A Spring Clear-Sky Day, 23 May 2015}
