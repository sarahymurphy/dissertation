\chapter{Improvements to the Weather Research and Forecasting Model over First-Year Sea Ice}

\noindent \textbf{Abstract}

\section{Introduction
}
\section{How does the WRF output (both offline and online) compare with the other formulations (bulk/MEP/EddyPro)?}
\begin{enumerate}
    \item Is MEP a better way to simulate the latent and sensible heat flux over the surface? What if we substituted this into the offline calculation?
\end{enumerate} 

\section{How sensitive is the WRF offline calculation to the universal functions?}

\begin{enumerate}
    \item Can we improve WRF just by implementing a different universal function relationship?
\end{enumerate}

\section{What other things in WRF should be changed to improve performance over first-year sea}
\begin{enumerate}
    \item Cloud properties, surface layer constants, ...
\end{enumerate}

\section{Discussion and Conclusions}



% from my prelim
The global climate is heavily influenced by processes that occur in the Arctic. A lack of observation has led to difficulty understanding atmospheric processes in polar regions \cite{Persson:2002ka}. The IPCC \cite{IPCC:14} lists high to very high confidence in observed changes in marine and terrestrial ecosystems, food production, and health/economics in the polar regions due to climate change. Both the Arctic and Antarctic are projected to continue warming more quickly than the global mean temperature. There is high confidence that year-round sea ice extent has been decreasing since before 1970 and will continue to decrease in the future. The IPCC \cite{IPCC:14} indicates that a changing climate may have large impacts on indigenous people, marine mammals, and coastal ecosystems in the Arctic. However, there are many observational uncertainties, such as in anthropogenic forcings in the polar regions, leading to low confidence \cite{IPCC:14}. Observational uncertainties in these regions stem from the harsh conditions and difficulty in deploying ground-based instruments.

Changes occurring in the Arctic modify the atmospheric circulation, impacting both cloudiness and radiation at the surface \cite{Zhang:2008cn}. According to Stroeve et al. \cite{Stroeve:2012dl}, many CMIP5 members show that there may be ice free conditions within the next couple of years. In addition, Maslanik et al. \cite{Maslanik:2007ha} states that the ice cover can be very sensitive to temperature changes, resulting in large and rapid changes in sea ice extent. Temperatures in the Arctic have increased much more quickly than those elsewhere on the planet, averaging 2 to 3 times faster than the global average, with longer melt seasons and more variability in snow cover \cite{Sledd:2019bz} \cite{AACI:05}. In addition, multi-year ice is disappearing, and being replaced with thin, single-year ice. This movement toward a new ice regime has been referred to as a shift toward a new climate state in the Arctic \cite{Verlinde:2007fu}.  

Arctic amplification is a significant factor in the poles warming more quickly than other parts of the globe. Arctic amplification modifies the surface energy budget. The amplification occurs because of as the Arctic warms, the albedo decreases. The ice-albedo feedback occurs as warmer temperatures cause an increase in ice melt. The albedo of the open ocean or bare land is much lower than that of sea ice or snow, resulting in greater absorption of incoming solar radiation. As the open ocean (or bare surface) absorbs more shortwave radiation, the ocean (land) surface heats, resulting in a greater extent of melt, and therefore further amplification of the ice-albedo feedback. Clouds are an important factor to consider when observing this feedback, as they can enhance or slow surface melt by modifying the energy balance. The presence of clouds can reduce the impacts of sea ice loss by reflecting shortwave radiation away from the earth due to their high albedo, therefore reducing the amount of radiation absorbed by the low albedo surface. Hwang et al. \cite{Hwang:2018jb} states that clouds can reduce the ice-albedo feedback at the surface by around 0.44 Wm$^{-2}$, as was seen in 2007, when a record low sea ice extent was observed \cite{Hwang:2018jb} \cite{Sledd:2019bz}.

The cloud radiation feedback is more dynamic than the ice-albedo feedback in the sense that it can be either positive or negative depending on a variety of influences. Clouds change change the surface temperature by modifying both the shortwave and longwave radiation reaching the surface. Cloud microphysics, such as phase and particle size, as well as macrophysics, including fractional coverage, cloud height and thickness, can influence the amount of radiation reaching the surface and, in turn, influence the impact of the cloud-radiation feedback \cite{Uttal:2002vw}. This relationship is nonlinear and depends on both cloud and sea ice characteristics \cite{Intrieri:2002cn}.

Clouds, as stated above, can modify both the radiation budget and impact the ice-albedo feedbacks. The influence of clouds is magnified by the high surface albedo and the lack of atmospheric moisture \cite{Shupe:2003vt}. Clouds have the greatest potential to modify heat exchange in the Arctic \cite{Intrieri:2002cn}. While we do have some estimates on how much the clouds can impact these feedbacks, more information is needed to quantify the exact influence. Sledd et al. \cite{Sledd:2019bz} states that the clouds are the most important driver in changes in top-of-atmosphere albedo over the entire globe, including at the poles, regardless of the high surface albedo. Cloud characteristics were shown to directly impact the ice thickness in studies by Curry et al. \cite{Curry:1992vn} and Beesley et al. \cite{Beesley:2007vg}.  

The impact of clouds is often quantified using cloud radiative forcing (CRF). The CRF describes how clouds modify the radiation at the surface by taking the difference between the observed radiation and the clear-sky radiation \cite{Ramanathan:1989tl}. When positive radiative forcing is observed, there is a surplus of net radiation at the surface, and warming occurs. When the CRF is negative, cooling occurs at the surface. During clear skies, CRF should equal zero, as the actual radiation should be the same as the estimation of clear-sky radiation. Clear-sky radiation is often calculated from a radiative transfer model or estimated using observed clear-sky times.

Net cloud radiative forcing is a balance of surface warming and cooling due to modifications in radiation as a result of cloud cover. Curry and Ebert \cite{Curry:1992vn} and Intrieri et al. \cite{Intrieri:2002cn} found that, in the Arctic, clouds warm the surface over the entire year (have a positive cloud radiative forcing) except for in mid-July, when the sun highest above the horizon. This nearly year-round warming is due to the small amount of shortwave radiation. When there is solar radiation present, the low sun elevation angle and high surface albedo reflects much of the shortwave radiation away. In addition, the low-level clouds are often emitting longwave radiation at warmer temperatures than the ice surface due to surface temperature inversions \cite{Shupe:2003vt}.

Models in the polar regions have the largest uncertainties relative to other parts of the Earth \cite{Holland:2003hg} \cite{AACI:05}. Models have difficulty simulating radiation accurately during times of thick clouds. A likely reason for this is the model’s inability to estimate the amount of liquid phase drops within the cloud \cite{Graham:2017cs}. A surface inversion often persists over the winter months and processes under these stable conditions are not well understood or modeled \cite{Tastula:2012fta}. In the summer, these inversions are often elevated compared to the wintertime surface inversions \cite{Serreze:1992vz}. Models, however, are integral in understanding the processes occurring in the poles, particularly the radiation.  Unfortunately, few field experiments collecting data for validation exist.

Reanalysis products are often used to study the Arctic climate. This poses a challenge as they are not as thoroughly verified as in other locations due to the extreme climate and dark winters preventing accurate, long-term, multi-season, in-situ and satellite measurements. Biases in clouds result in difficulties in resolving the surface energy budget. Recent studies have shown that when compared with surface observations, reanalysis have large biases in cloud properties (liquid/ice water path, fraction). A number of field experiments have shown that mixed-phase clouds are dominant in autumn through spring in the lower levels at high latitudes \cite{Intrieri:2002cn} \cite{Wang:2005vx}. Cloud micro- and macrophysics are closely tied into the surface energy budget, but parameterizations of such are not well developed in models. Particularly in the Arctic, the radiative properties of clouds and how they are parameterized in models is of importance to modeling the surface energy budget. 

The  Norwegian Young Sea Ice Experiment (N-ICE, described in section 3) field campaign was the first winter field experiment in the Arctic since the Surface Heat Budget of the Arctic (SHEBA) experiment, taking place on-board a research vessel frozen into Arctic sea ice. SHEBA’s primary goals were to observe the surface energy budget, ice mass balance, and ocean-ice-atmosphere interactions in the Arctic during a year-long period from October 1997 to October 1998. Much like N-ICE, SHEBA was motivated by changes in the Arctic and the need for a better understanding of physical processes in the polar regions \cite{Randall:1998up}. A secondary objective of SHEBA was to improve model simulations of the Arctic for use in global climate models \cite{Uttal:2002vw}. Both the ice-albedo feedback and the cloud-radiation feedback were extensively studied using datasets collected during this field experiment. However, this experiment occurred 18 years prior to N-ICE and in a different location of the Arctic, influenced by different synoptic conditions. 

During SHEBA, the Canadian research ship, Des Groseilliers, was allowed to drift with the sea ice. A map of the ship locations in both SHEBA and N-ICE is shown in Figure 1. SHEBA took place across in the Beaufort Sea, with the entire experiment taking place north of Alaska \cite{Uttal:2002vw}. In contrast, the N-ICE campaign took place north of Svalbard. The sea ice during SHEBA was thicker than that during N-ICE, as the experiment was conducted further from the ice edge \cite{Graham:2017cs}. 

During the start of SHEBA, the western Arctic had an anomalously large amount of multi-year ice. In addition, the autumn upper ocean had a lower salinity and warmer temperature than expected, indicating a larger ocean heat flux than was typical of the area during the summer resulting in larger melt due to the reduction in sea ice cover the previous year. Comparing SHEBA data to estimates from other fields experiments showed that during transition seasons (September, October, November, March, and April), SHEBA had larger incoming longwave radiation by 2 to 45 Wm$^{-2}$ than other studies. This could be caused by either an increase in the number of warm air masses over SHEBA or an increase in cloud cover \cite{Persson:2002ka}. SHEBA was an important field experiment that filled many gaps in our understanding of cloud and radiation processes. One of the most important findings from SHEBA was that, even at temperatures well below freezing, mixed-phase clouds occurred often. 