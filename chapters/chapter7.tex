\chapter{Conclusions, Recommendations, and Future Work}
\vspace{1 cm}
\begin{spacing}{1} \begin{quote} 
\noindent \emph{The polar regions, notably the Arctic and maritime Antarctic, are experiencing impacts from climate change at magnitudes and rates that are among the highest in the world, and will become profoundly different in the near-term future (by 2050) under all warming scenarios (high confidence).}\end{quote}
\hspace{6 cm} - IPCC Sixth Assessment Report, August 2021  
\end{spacing}
\doublespacing
\section{Conclusions}

This chapter provides conclusions of this study and makes recommendations for improving Polar WRF over thin, first-year sea ice. In addition, future work is outlined for improving the Polar WRF. 

The results from the Polar Weather Research and Forecasting model (Polar WRF) in Chapter 3 showed that, regardless of cloud microphysics (CM) or planetary boundary layer (PBL) scheme, the model had some large biases. In the winter, every model simulation had a negative sensible heat bias, and all but one (the P3-YSU simulation) had a positive bias in longwave radiation. In the spring, all model simulations underestimated the net longwave radiation and overestimated the net shortwave. Latent heat flux biases were larger in the spring than in winter. The 2-MYNN run had the lowest overall biases in the winter and the G-YSU had the lowest biases in the spring and summer. Cloud radiative forcing (CRF) for these two schemes was calculated in Chapter 4 and showed that clouds in the 2-MYNN scheme had the correct CRF, but in the spring, the modeled CRF was quite low compared to the observations, due to the fact that the cloud fraction was greatly underestimated by Polar WRF.

Eddy covariance measurements from N-ICE were processed through the LI-COR’s EddyPro 7 software to estimate surface heat fluxes over the first-year ice using the eddy covariance method. The Noah Land Surface Model (LSM) used to calculate surface fluxes in Polar WRF currently uses the bulk approach over sea ice, which requires estimations of the Monin-Obukhov scaling and stability parameters. These parameters are dependent on wind, potential temperature gradients, and surface stability. The cold surface often results in a strongly stable surface layer, a characteristic typical in the polar regions, presenting potential errors in the stability regimes used to estimate these parameters. In Chapter 4, latent and sensible heat flux values were calculated for N-ICE using the bulk flux algorithm and the Maximum Entropy Production (MEP) method. Chapter 5 showed that both techniques of calculating the turbulent fluxes did an acceptable job of estimating the sensible heat flux, but both have larger ranges of values for the latent heat flux, indicating these equations were producing too much phase change. Chapter 6 compares these equations to the equations used by Polar WRF. The equations within WRF do very well at estimating the turbulent fluxes when the correct radiative fluxes are specified, indicating that most of the error within Polar WRF are due to inaccurate simulations of cloud properties. 

Chapter 6 presents the model sensitivity to changing land use parameters and to different turbulent flux equations. These results suggest that the model performance can be improved over first-year sea ice by modifying key parameters provided by the lookup tables within the model. 

\section{Recommendations}
This section suggests recommendations for improving the Polar WRF model, based on the sensitivity studies performed in Chapter 6 that were motivated by Chapters 3 and 4.

\subsection{Setting Surface Parameters}
More testing needs to be done to determine if these values are best under specific conditions, but using the results described above, the following modifications are recommended for the LANDUSE.TBL file when working over first-year sea ice. However, it should be stressed for the user to be mindful that cloudiness is extremely important to the surface energy budget, and to be aware that these values may produce less cloud cover than using the original USGS values found in the LANDUSE.TBL. Therefore, additional modifications will likely be necessary to the CM schemes.

\begin{enumerate}
  \item Increase albedo in both the winter and summer (column 3, Table \ref{tab:wrf:recommendations})
   \item Decrease surface roughness from 0.1 to 0.001 (column 5, Table \ref{tab:wrf:recommendations})
  \item Decreased surface heat capacity and use a different value for winter and summer (column 6, Table \ref{tab:wrf:recommendations})
\end{enumerate}

\subsection{Turbulent Flux Equations}
Latent and sensible heat flux values in WRF are calculated within the surface modules using meteorology from the BL scheme and radiation from the radiation scheme. In this study, all physics schemes except for the surface layer (SL) scheme were removed and were run "offline", using the idealized version of WRF at a single location to calculate sensible and latent heat fluxes. When observed radiation, wind, temperature, and moisture values were used as input to the offline model, the calculated sensible heat flux values were quite accurate compared to the N-ICE measurements. The latent heat fluxes, however, were not and often had significantly larger negative values than what was observed. The latent heat flux at N-ICE could not be accurately replicated by the offline WRF calculation or any of the external ways of calculating it. Chapter 3 shows that all WRF simulations struggled to replicate the latent heat flux. Chapter 5 describes several methods for calculating the latent heat flux, but none accurately represented the latent heat flux seen at N-ICE.

Chapter 3 showed that the Polar WRF model, when run as a complete model, does not accurately represent the surface conditions seen at N-ICE. The sensible heat flux had a mean bias over all model simulations of -20 $Wm^{-2}$ in the winter and 4.9 $Wm^{-2}$ in the spring. Latent heat flux had a mean bias of -0.6 $Wm^{-2}$ in the spring and 3.3 $Wm^{-2}$ in the spring. However, the offline calculations accurately simulated the sensible heat flux, indicating that some input parameters used in the SL scheme and LSM are the primary source of this error. Radiation is input to the surface from the radiation scheme in WRF but came directly from the N-ICE observations in the offline calculation. The offline calculation allows us to isolate the surface schemes and ensure any cloud-related errors are not impacting errors in the surface fluxes. The radiation is almost certainly incorrect as a result of incorrect cloud properties in the model. As described in Chapter 4, the clouds observed at N-ICE were primarily mixed phase (mixture of water and ice) and, as shown in Chapter 3, they were not accurately captured in any of the WRF simulations. 

\section{Future Work}
\subsection{Flux Equations}
Two ways of estimating the latent and sensible heat flux were presented in Chapter 5 and were compared to equations used within WRF in Chapter 6. This research could be expanded to other ways of calculating the latent and sensible heat flux, such as the Priestly-Taylor method or the Bowen Ratio method. The two methods selected here were chosen due to their use in models (the bulk algorithm method) and recent development for use in the polar regions (Maximum Entropy Production method). However, there is the potential for other methods of estimating surface fluxes to perform well over first-year sea ice.

\subsection{Clouds Microphysics Schemes}
An early goal of this research was to look further into the cloud microphysics (MP) schemes used in WRF to improve polar clouds, particularly mixed-phase or those with small ice particles. However, upon further investigation of the model, there were very large biases in the latent and sensible heat flux calculations. While a portion of this is attributed to the incorrect portrayal of clouds, the amount cannot be determined without first finding the errors in the flux equations themselves. While the recommendations made here may improve the atmospheric structure and moisture exchange within the model, they do not fix issues with the CM schemes. 

All WRF microphysics (MP) schemes with the exception of the Predicted Particle Phase (P3) scheme use``binning" to identify particle characteristics. This means that cloud particle properties are simplified and assigned to a range of particle sizes. This might be an issue when mixed-phase particles are present. The P3 scheme removes this simplification by evolving the assigned cloud particle properties with the evolution of the particle itself, but has a size cutoff and, therefore, has the potential to exclude smaller ice particles that are often observed in the polar regions. Chapter 3 showed that the P3 scheme did not simulate the atmosphere at N-ICE well, and this could be due to an inappropriate size cutoff for cloud particles in the polar regions. Reducing this size limit could improve cloud properties from runs using the P3 scheme.

