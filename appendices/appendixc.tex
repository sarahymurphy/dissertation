\chapter{Additional Polar WRF Validation Statistics}

\begin{table}[p]
\center
\centering
\footnotesize
\doublespacing
{
\begin{tabular}{| c | c | c | c | c |}
\hline\rowcolor[HTML]{F3F3F3} & \textbf{Winter Net} & \textbf{Spring LW} & \textbf{Spring SW} & \textbf{Spring Net} \\
\hline
\rowcolor[HTML]{F0F8E6}\textbf{G-YSU} & -40.9, (16.5), 43.8
& -40.9, (16.5), 43.8 & 17.0, (-7.1), 23.0 & -23.8, (9.4), 28.2	\\
\rowcolor[HTML]{E0EDF4}\textbf{G-MYJ} & -39.4, (17.4), 43.0
& -39.4, (17.4), 43.0 & 16.1, (-7.7), 22.7	& -23.3, (9.7), 28.0 \\
\rowcolor[HTML]{FEEEF5}\textbf{G-MYNN} & -32.8, (23.5), 38.0
& -32.8, (23.5), 38.0 & 14.8, (-8.5), 21.3 & -18.0, (15.0), 24.0 \\
\rowcolor[HTML]{E0EDF4}\textbf{5-MYJ} &	-34.8, (22.8), 39.3
& -34.8, (22.8), 39.3 & 17.5, (-5.7), 22.3 & -17.3, (17.0), 26.7 \\
\rowcolor[HTML]{FEEEF5}\textbf{5-MYNN} & -25.2, (31.1), 32.1
& -25.2, (31.1), 32.1 & 14.1, (-8.9), 20.0 & -11.1, (22.2), 23.3 \\
\rowcolor[HTML]{F0F8E6}\textbf{P3-YSU} & -26.2, (30.9), 39.5
& -26.2, (30.9), 39.5 & 13.6, (-10.3), 21.2 & -12.6, (20.5), 28.1\\
\rowcolor[HTML]{E0EDF4}\textbf{P3-MYJ} & -39.4, (17.4), 43.0
& -39.4, (17.4), 43.0 & 16.1, (-7.7), 22.7 & -23.3, (9.7), 28.0 \\
\rowcolor[HTML]{E0EDF4}\textbf{2-MYJ} & -13.2, (41.7), 24.2
& -13.2, (41.7), 24.2 & 8.1, (-14.2), 16.6 & -5.0, (27.5), 20.9 \\
\rowcolor[HTML]{FEEEF5}\textbf{2-MYNN} & 1.6, (55.8), 20.2 
& 1.6, (55.8), 20.2 & 65.9, (-16.4), 16.4 & 7.5, (39.4), 19.2 \\
\hline
\end{tabular}
\caption[Polar WRF CRF mean modeled bias, flux, and absolute error.]{Mean model CRF bias (left), mean flux (middle, in parentheses), and mean absolute error (right) ($Wm^{-2}$). No shortwave CRF values are presented for winter as no shortwave radiation was present until spring. Floe 1 and 2 are considered winter and Floe 3 and 4 are spring. Acronyms represent the MP and PBL schemes (defined in Table \ref{tab:schemes}). Rows are ordered by MP scheme and colored by PBL scheme.}
\label{tab:wrfstats_CRF}
\end{table}}