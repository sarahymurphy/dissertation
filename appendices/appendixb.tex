\chapter{Description of the Norwegian Young Sea Ice Field Campaign}

% this is from my prelim
The Norwegian Young Sea Ice Experiment was conducted during the six-month transition from the winter to summer (January to June) in 2015 in the Arctic ocean north of Svalbard. All instruments were either deployed on-board the Norwegian research vessel \textit{Lance} or on the sea ice nearby the ship. The ship and surrounding camp can be seen in Figure 2.

Measurements of the energy balance and cloud properties (fraction, height, microphysical, and temperature) can give important insight to climate processes and radiative transfer, but are rarely measured together \cite{persson:2002} \cite{schweiger:2004}. The Norwegian Young Sea Ice Experiment (N-ICE2015) is the first experiment to study all of these factors during both winter and summer since SHEBA in 1997 and 1998 \cite{walden:2017}. 

There were several periods of low atmospheric pressure and increased wind speed that were defined as storms. These storms were also accompanied by increases in the integrated water vapor and changes in wind direction \cite{kayser:2017}. Storms are further described in Cohen et al.  \cite{cohen:2017} and Kayser et al. \cite{kayser:2017}. More details about the experiment and datasets collected can be found in Granskog et al. \cite{Granskog:2016dn}. There are three gaps in the dataset resulting from ice breakup. During the experiment, the floe that the ship was anchored to broke apart, resulting in the inability to continue measurements on  the surrounding sea ice. During these times, the instruments were removed from the sea ice, and the ship then sailed to another floe further north on which the instruments were redeployed. Itkin et al. \cite{itkin:2017} describes the proximity to the sea ice edge throughout the experiment. During the majority of the experiment, the ship was stationed between 50 and 250 km from the ice edge. 

A variety of instruments were deployed during the N-ICE campaign that will be used in this project, including  radiosondes, a MicroPulse Lidar (MPL), a meteorological tower, an Eddy Covariance system, and broadband shortwave and longwave radiometers. 

Vaisala RS92-SGP radiosondes were launched from the ice surface (first floe) or the ship deck (following three floes) twice daily around 1100 and 2300 UTC. They recorded temperature, relative humidity, wind speed and direction, pressure, and geopotential height as high as 30 km. Data are recorded by the radiosondes on a two-second time interval and transmitted to the ground using the Vaisala MW31 ground station \cite{kayser:2017} \cite{cohen:2017}. More information and analysis of the radiosondes can be found in Kayser et al. \cite{kayser:2017}.

Data from the MPL were recorded every 14 seconds up to a height of 20 km. The MPL records backscattered light from clouds and operates at 532 nm. The range resolution is 15 m, with a 18 km maximum cloud base height. Signature distance uncertainties are ± 2\% due to timing uncertainties within the instrument. This instrument is more sensitive to water particles than ice, so some cloud types may be biased toward higher percentages of water than ice within the cloud. The MPL is easily attenuation by optically thick clouds. In some instances when a low water cloud is detected, it is possible that more cloud layers exist above this layer that can not be measured by the MPL. 

A meteorological tower was deployed on the ice 300 to 400 m away from the ship. This tower was set up within a few days of anchoring to each new floe and recorded relative humidity and temperature (Vaisala HMP155), pressure (RM Young 61302 V), and wind speed and direction (Lufft Ventus V200A-UMB) at 2 m, 4 m, and 10 m heights. All measurements were collected by a Campbell Scientific CR30000 data logger at 1 second resolution. Periods of missing tower data were reconstructed using temperature and wind information from the ship (sensors mounted 22 to 24 m above the surface). More information about the meteorological measurements, temperature and wind reconstruction using the ship data, a diagram of the meteorological tower set up, and a comparison of the meteorology to SHEBA can be found in Cohen et al. \cite{cohen:2017}.

Radiometers (Kipp and Zonen CMP22 and CGR4) were set up 1 to 1.2m above the surface near the meteorological tower to measure upward and downward components of longwave and shortwave components of radiation. Kipp and Zonen CVF4 ventilation units were used to heat and ventilate the radiometers. More information about the radiometers and an analysis of the surface energy budget can be found in Walden et al. \cite{walden:2017}.

Turbulent flux data was collected by a closed path EC flux system (Campbell CPEC200) at a varying frequency (10 or 20 Hz). This system contains a sonic anemometer and a closed path, infrared gas analyzer. These allow observation of the heat and moment exchanges and the water vapor and carbon dioxide mixing ratios, respectively. This system was set up next to the meteorological tower over a snow covered surface. Further information about the EC Flux system can be found in Walden et al. \cite{walden:2017}.

\section{Atmospheric components of the surface energy budget over young sea ice: Results from the N-ICE2015 campaign, Walden et al. (2017)}

This paper detailed the turbulent and radiative fluxes over the thin sea ice. Surface and atmospheric conditions were also covered. Snow albedo was around 0.85 in the winter and between 0.72 and 0.80 in the spring and summer. Stable stability was found in the winter, followed by unstable in the spring and approximately neutral in the summer (once 0°C skin temperature was reached). Negative average radiative and turbulent heat fluxes occurred in the winter, ranging between 40 to $0 Wm^{-2}$. In the summer, positive values of as high as $60 Wm^{-2}$ were recorded. Winter sensible heat flux ranged from $20$ to $30 Wm^{-2}$ and spring and summer from 0 to $-20 Wm^{-2}$. Positive values indicate flux into the surface.

My contribution to this publication was to fix several problems with the flux dataset collected during the N-ICE campaign and to process the data through the EddyPro software. The most restrictive data problem was the number of data gaps throughout 30-minute data files, which was caused by a programming error in the datalogger. In some cases, the amount of missing data made the file unable to be processed. To fix this, the data were filled by taking the section of data before it (or after, in the case that the missing data was too close to the start of the dataset) and replicating it for the time period with no recorded data. To ensure that this method of data filling was acceptable, data from Barrow, Alaska was used to compare the post-processed data of a complete dataset with the post-processed results from the same dataset after (artificially added) gaps had been filled. Analysis of both the difference in sensible heat fluxes and the turbulent spectra from before and after the data filling were examined and determined the filling method appropriate for the type of gaps in the N-ICE data. 