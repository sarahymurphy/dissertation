The Arctic is warming more quickly than other regions, resulting in modifications to the surface energy budget. The Norwegian Young Sea Ice (N-ICE) field campaign took place in January through June 2015, monitoring key components of the surface energy budget. This was the first field campaign to take observations of the surface energy budget during the seasonal transition from winter to spring/summer since the Surface Heat Budget of the Arctic (SHEBA) field campaign in 1997 through 1998. These observations are valuable not only due to the amount of time that has passed since the SHEBA field experiment but also because N-ICE took place on younger, thinner sea ice, and in a different part of the Arctic. This study uses data from the N-ICE experiment to recommend improvements to the Polar Weather Research and Forecasting (WRF) model. The goals of this study are: 1) compare atmospheric measurements to models, 2) investigate model cloud microphysics and boundary layer parameterizations within WRF, 3) determine what cloud properties were observed during N-ICE, 4) investigate the effects of turbulent fluxes and clouds on young sea ice, and 5) make recommendations for improving the Polar WRF model. The Polar WRF model results show that there were issues in calculating both the turbulent and radiative fluxes over first-year sea ice. The cloud conditions during N-ICE varied, and mixed-phase clouds (a mixture of water and ice) were seen throughout the entire experiment. The model, however, did not accurately simulate the radiative impacts of these clouds. To determine if the calculations of sensible and latent heat flux can be accurately estimated given the correct cloud conditions, a study was conducted comparing turbulent fluxes from the Maximum Entropy Production method and the bulk flux algorithm (based on Monin-Obukhov theory and the technique used to estimate fluxes within WRF). Sensible heat flux was accurately estimated, but all equations overestimated the amount of latent heat flux. When compared to an idealized (offline) version of the WRF model using the measured radiative fluxes as input, the WRF model estimates the turbulent fluxes well, indicating that the model is using an appropriate version of the flux equations and that most of the error within the model can be attributed to bias in the cloud properties.